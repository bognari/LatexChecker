\section{Aufgabenstellung}
\begin{frame}
\frametitle{Aufgabenstellung -- Projektidee}
\begin{block}{Aufgabenstellung:}
    Entwicklung eines Tools zur Überprüfung von Latex-Quellcode vergleichbar dem Checkstyle für Java
\end{block}
\begin{block}{Ziel:}
    \begin{itemize}
        \item Entwicklung eines Tools zur Überprüfung von LaTeX-Quellcode auf Einhaltung stilistischer, sprachlicher und syntaktischer Kriterien
        \item Kriterien sollen durch Regeln in logischer Form, regulären Ausdrücken oder Automaten angegeben werden
    \end{itemize}
\end{block}
\end{frame}
\begin{frame}
\frametitle{Aufgabenstellung -- Organisation des Projektes}
\begin{block}{\vspace*{-3ex}}
	\begin{itemize}
	  	\item Problem zu Beginn: Stundenpläne der Mitglieder
	  	\begin{itemize}
	  		\item Keine festen Termine
	  		\item Einfluss agiler Methoden
	  	\end{itemize}
	  	\item Kommunikation innerhalb der Gruppe mit Hilfe der üblichen Medien
	  	\item Treffen zum Arbeiten und Besprechungen in der Uni
	\end{itemize}
\end{block}
\end{frame}
\begin{frame}
\frametitle{Aufgabenstellung -- Verwendete Tools}
\begin{block}{\vspace*{-3ex}}
	\begin{itemize}
	 	\item Ant, später Wechsel auf Maven
	  	\item Eclipse Kepler
	  	\item IntelliJ IDEA 13
	  	\item \LaTeX-Distribution
	  	\item Online-Tool RegExr
	  	\item svn 
		%\item LanguageTool (www.languagetool.org)
	\end{itemize}
\end{block}
\end{frame}
\begin{frame}
\frametitle{Aufgabenstellung -- Wechsel auf Maven}
\begin{block}{\vspace*{-3ex}}
	\begin{itemize}
		\item Geringerer Administrationsaufwand hinsichtlich Konfiguration
		\item Ant hier fehleranfälliger
		\item bessere Abhängigkeitsverwaltung
		\item $\Rightarrow$ wechsel der anderen Teammitglieder auf Itellij, da Maven hier besser integriert ist.
	\end{itemize}
\end{block}
\end{frame}