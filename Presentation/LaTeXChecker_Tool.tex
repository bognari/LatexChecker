\begin{frame}
\frametitle{Systemvoraussetzungen}
\begin{block}{\vspace*{-3ex}}
	\begin{itemize}
	  %\item Installation einer aktuellen LaTeX-Distribution wie MiKTeX
	  \item Benutzung: 
	  \begin{itemize}
	  	\item Java-Laufzeitumgebung ab Version 7 SE reicht aus
	  	\item \emph{empfohlen}: Installation einer aktuellen LaTeX-Distribution wie MiKTeX  
	  \end{itemize}
	  \item Modul Entwicklung: 
	  	\begin{itemize}
	  	  	\item Java-SDK ab Version 7
	  	  	\item Maven
	  	  	\item \emph{empfohlen}: IDE wie Intellij mit Maven Unterstützung
	  	\end{itemize}  	  
	\end{itemize}
\end{block}
\end{frame}

\begin{frame}
\frametitle{Aufrufparameter}
\begin{block}{\vspace*{-3ex}}
\begin{table}[h]
\caption{Informationsparameter}
	\begin{tabular}{l|p{9cm}}
		-dc & erzeugt eine Standart Konfiguration beim angegeben Pfad\\
		-h  & gibt einen Hilfetext für alle Aufrufparameter aus\\
		-ver & zeigt Informationen zur Version, benutzten Software und Lizens an
	\end{tabular}
\end{table}
\end{block}
\end{frame}

\begin{frame}
\frametitle{Aufrufparameter}
\begin{block}{\vspace*{-3ex}}
\begin{table}[h]
\caption{Standard Parateter}
	\begin{tabular}{l|p{9cm}}
		-t & gibt den Pfad zur Latex Datei an\\
		-cs & gibt den benutzten Zeichensatz in der Latex Datei an\\
		-c & die zuladenen Konfigurations Dateien\\
		-l & gibt die verwendete Sprache an
	\end{tabular}
\end{table}
\end{block}
\end{frame}

\begin{frame}
\frametitle{Aufrufparameter}
\begin{block}{\vspace*{-3ex}}
\begin{table}[h]
\caption{Erweiterte Parateter}
	\begin{tabular}{l|p{9cm}}
		-abb & gibt an, ob Abkürzungen maskiert werden soll oder nicht, bei Deaktivierung, wird die standart Satzerkennung aus dem LanguageTool benutzt\\
		-nd & wertet die ganze Datei und nicht nur den Inhalt in der \emph{document} Umgebung aus\\
		-uc & gibt an, ob beim Einlesen schon Latex Befehle verarbeitet werden sollen, dies ist die einzige Möglichkeit um ein \command{"a} in ä um zu wandeln
	\end{tabular}
\end{table}
\end{block}
\end{frame}

\begin{frame}
\frametitle{Konfuguration}
\begin{block}{\vspace*{-3ex}}
\begin{itemize}
\item Format: Json
\item Mehre Konfiguratonsdateien möglich
\item Standard Konfiguration (config.json) ist in der Jar im conf Ordner enthalten und wird immer mit benutzt
\item Jede Konfiguration erweitert die standard Konfiguration mittels merge
\item Somit können die Kunfugurationen sich gegenseitig ergänzen 
\end{itemize}
\end{block}
\end{frame}

\begin{frame}[fragile]
\frametitle{Konfiguration Standard}
\begin{block}{\vspace*{-3ex}}
\begin{lstlisting}
{	"ABBREVIATIONS": {
        "language" : ["abbreviation"]},
    "PARTS": {"part" : level},
    "ENVIRONMENT_WHITELIST": ["environment"],
    "MATH_ENVIRONMENTS": ["environment"],
    "ITEM_ENVIRONMENTS": ["environment"],
    "COMMAND_WHITELIST": {"command" : [arg]},
    "LATEX": {"command" : "translation"},
    "CONVERTING": {"command" : "translation"},
    "ENVIRONMENT_BREAK_LATEX": ["environment"],
    "COMMAND_BREAK_LATEX": ["command"]}
\end{lstlisting}
\end{block}
\end{frame}

\begin{frame}[fragile]
\frametitle{Konfiguration}
\begin{block}{\vspace*{-3ex}}
\begin{itemize}
\item Jede Regel wird durch ein Modul umgesetzt
\item Die meisten Module können für verschiedene Regeln konfiguriert werden
\item Jede Regel muss einen eindeutigen Namen besitzen und ein Modul
\item Falls benötigt, können auch andere Einstellungen ergänzt werden
\end{itemize}
\end{block}
\end{frame}

\begin{frame}[fragile]
\frametitle{Konfiguration}
\begin{block}{\vspace*{-3ex}}
\begin{lstlisting}
{	"RULES" : {
        "rule name" : {
            "Class" : "class path to the module",
            "other" : ...
        }
    }
}
\end{lstlisting}
\end{block}
\end{frame}

\begin{frame}
\frametitle{Modul -- TextRegex}
\begin{block}{\vspace*{-3ex}}
\begin{itemize}
\item Als Beispiel betrachten wir das Modul TextRegex
\item Testen den Text mit Regulären ausdrücken
\item Der Text und die Ausdrücke sind genau festlegbar
\item Für jede Quelle (Text) muss eine neue Regel erstellt werden, da nur eine Quelle pro Regel erlaubt ist
\end{itemize}
\end{block}
\end{frame}

\begin{frame}[fragile]
\frametitle{Modul -- TextRegex}
\begin{block}{\vspace*{-3ex}}
\begin{table}[h]
	\begin{tabular}{l|p{5cm}}
		\lstinline|"CaseSensitive" = bool| & Gibt an, ob nach Groß- und Kleinschreibung unterschieden werden soll\\
		\lstinline|"HeadlineFrom" = number| & Startlevel für die Suche in der Dokumenten Hierarchie\\
		\lstinline|"HeadlineTo" = number| & Endlevel für die Suche in der Dokumenten Hierarchie\\
		\lstinline|"ListEndWith" = "Text"| & Ein Text, der nach einem Wort aus der Wortlist gematcht wird\\
		\lstinline|"ListStartWith" = "Text"| & Ein Text, der vor einem Wort aus der Wortlist gematcht wird
	\end{tabular}
\end{table}
\end{block}
\end{frame}

\begin{frame}[fragile]
\frametitle{Modul -- TextRegex}
\begin{block}{\vspace*{-3ex}}
\begin{table}[h]
	\begin{tabular}{l|p{5cm}}
			\lstinline|"Msg" = "Text"| & Der Nachrichten Prototyp\\
			\lstinline|"OnlyAtEnd" = bool| & Gibt an, ob nur am Ende (\$) gematcht wird\\
			\lstinline|"OnlyAtStart" = bool| & Gibt an, ob nur am Anfang (\textasciicircum) gematcht wird\\
			\lstinline|"Source" = "Text"| & möglich sind: text, environment, command, headline, bullet, tex oder latexText\\
			\lstinline|"SourceList" = ["Text"]| & Die Eingabemöglichkeiten für SourceList sind von "Source" abhängig (für Regex setzte UseRegex auf true)
	\end{tabular}
\end{table}
\end{block}
\end{frame}

\begin{frame}[fragile]
\frametitle{Modul -- TextRegex}
\begin{block}{\vspace*{-3ex}}
\begin{table}[h]
	\begin{tabular}{l|p{5cm}}
			\lstinline|"UseEscaping" = bool| & Gibt an, alle eingaben escapt werden mit \textbackslash Q Text \textbackslash E\\
			\lstinline|"UseRegex" = bool| & Gibt an, ob Regex in Listen benutzt wird\\
			\lstinline|"WordList" = ["Text"]| & Eine Liste von Wörtern, von denen eins gematcht wird (logisches Oder)
	\end{tabular}
\end{table}
\end{block}
\end{frame}

\begin{frame}[fragile]
\frametitle{Weitere Module}
\begin{block}{\vspace*{-3ex}}
\begin{table}[h]
	\begin{tabular}{l|p{7cm}}
			ChapterCheck & Testet auf: zu viele / wenige Unterkapitel, zu tiere Verschachtelung\\
			EnvTextCheck & Testet auf: zu wenig Text zwischen, vor (seit Beginn des Abschnits) und nach (bis Ende des Abschnitts) einer Umgebung\\
			HeadlineRegEx & Testet auf: Paragraphentitel enden immer mit Punkt oder Doppelpunkt, Keine Doppelpunkte in Überschriften und Grafik-/Tabellenunterschriften, Mindestens n Wörter in Überschriften und Grafik-/Tabellenunterschriften
	\end{tabular}
\end{table}
\end{block}
\end{frame}

\begin{frame}[fragile]
\frametitle{Weitere Module}
\begin{block}{\vspace*{-3ex}}
\begin{table}[h]
	\begin{tabular}{l|p{7cm}}
			BadLatex & Testet auf: nicht zu verwendenen Befehlen / Umgebungen\\
			ItemCheck & Testet auf: zu viele oder zu wenige Items in einer Umgebung\\
			LabelCheck & Testet auf: Jede Grafik ist min. 1 mal referenziert, testet, ob jede Umgebung (die angegeben ist) ein Label hat sowie ob jedes im Dokument existierende label auch referenziert wird.
	\end{tabular}
\end{table}
\end{block}
\end{frame}

\begin{frame}[fragile]
\frametitle{Weitere Module}
\begin{block}{\vspace*{-3ex}}
\begin{table}[h]
	\begin{tabular}{l|p{7cm}}
			LanguageToolChecker & Testet auf: Rechtschreibung und weitere Regels des LanguageTools\\
			ParagraphLength & Testet auf: zu viele oder zu wenige Sätze pro Paragraph\\
			TextLength & Testet auf: Satzlänge (Wort-, Zeichenanzahl)\\
			AbbreviationsCheck & Testet auf: falsche Abkürzungszwischenräume 
	\end{tabular}
\end{table}
\end{block}
\end{frame}