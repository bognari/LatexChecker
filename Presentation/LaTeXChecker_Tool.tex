\begin{frame}
\frametitle{\LaTeX~- StyleChecker -- Systemvoraussetzungen}
\begin{block}{\vspace*{-3ex}}
	\begin{itemize}
	  %\item Installation einer aktuellen LaTeX-Distribution wie MiKTeX
	  \item Benutzung: 
	  \begin{itemize}
	  	\item Java-Laufzeitumgebung ab Version 7 SE reicht aus
	  	\item \emph{empfohlen}: Installation einer aktuellen LaTeX-Distribution wie MiKTeX  
	  \end{itemize}
	  \item Modul Entwicklung: 
	  	\begin{itemize}
	  	  	\item Java-SDK ab Version 7
	  	  	\item Maven
	  	  	\item \emph{empfohlen}: IDE wie Intellij mit Maven Unterstützung
	  	\end{itemize}  	  
	  %\begin{itemize}
	  %	\item Wenn eigene Module entwickelt werden sollen: JDK ab Version 7
	  %\end{itemize}
	  %\item MacOSX, Windows ab XP, Linux
	\end{itemize}
\end{block}
\end{frame}
%\begin{frame}
%\frametitle{\LaTeX~- StyleChecker -- Aufbau und Funktion}
%\begin{block}{\vspace*{-3ex}}
%	\begin{itemize}
%	 	\item Core
%	  	\item Modules
%		\item resources
%	  	\item Test
%		\item Die zu überprüfenden Dateien
%	\end{itemize}
%\end{block}
%\end{frame}

% wie benutzt man das tool
% komandozeilen parameter beim starten
% lesen der "ausgaben" der einzelnen module
% die config dateien
% was sind module
% konfiguration des programms und der module über eine json datei (name ist eh egal)
% vorstellen der einzelnen module (regeln)
%
%Usage: <main class> [options]
%  Options:
%    -cs, --charset
%       Charset of the tex file
%       Default: UTF-8
%    -c, --config
%       Path to the config files
%       Default: []
%    --default, -dc
%       Create a default config at the given path
%       Default: <empty string>
%    --help, -h
%       Show all CLI options
%       Default: false
%    -l, --language
%       Language of the tex file
%       Default: en
%    -nd, --noDocument
%       Not limited to the "Document" environment
%       Default: false
%    -s, -t, --source, --tex
%       Path to the tex file
%       Default: <empty string>

