\section{Funktionsweise}
\begin{frame}
\frametitle{\LaTeX~- StyleChecker -- DocumentTree}
\begin{block}{\vspace*{-3ex}}
	\begin{itemize}
		\item Aufbau eines Dokuments in einer Baumstruktur
		\item Jedes Level eines Knoten entspricht der jeweiligen Hierarchie im Dokument in einem Abschnitt
		\item Jeder Kindesknoten ist wieder ein Baum
		\item Rootknoten als Beginn eines Dokuments
	\end{itemize}
\end{block}
\end{frame}
\begin{frame}
\frametitle{\LaTeX~- StyleChecker -- LanguageTool}
\begin{block}{\vspace*{-3ex}}
	\begin{itemize}
		\item Externe Bibliothek zur sprachlichen Überprüfung der \LaTeX~-Dateien
		\item Mögliche Sprachen: alle unterstützten Sprachen \\ z.B.: Deutsch, Englisch
	\end{itemize}
\end{block}
\end{frame}
\begin{frame}
\frametitle{\LaTeX~- StyleChecker -- Umsetzung der Regeln}
\begin{block}{\vspace*{-3ex}}
	\begin{itemize}
		\item Mit Hilfe von regulären Ausdrücken
		\begin{itemize}
			\item Erkennen von beispielsweise Sätzen, Abkürzungen und Überschriften
		\end{itemize}
		\item Zum Testen der regulären Ausdrücke: Einsatz von regexr.com
		\item Zugriff auf die einzelnen Klassen über eine API-Klasse
	\end{itemize}
\end{block}
\end{frame}