\section{Funktionsweise}
\begin{frame}
\frametitle{\LaTeX~- StyleChecker -- DocumentTree}
\begin{block}{\vspace*{-3ex}}
	\begin{itemize}
		\item Aufbau eines Dokuments in einer Baumstruktur
		\item Jeder Knotenlevel ein Abschnitt des Dokumentes, der von seinen Kindern untergliedert wird
		\item Jeder Knoten wieder ein Tree
		\item Rootknoten als Beginn eines Dokuments
	\end{itemize}
\end{block}
\end{frame}
\begin{frame}
\frametitle{\LaTeX~- StyleChecker -- DocumentTree}
\begin{block}{\vspace*{-3ex}}
	\begin{itemize}
		\item Abschnitte eines Dokuments zum Anwenden von Regeln 
		\item Level, Überschriften, der reine Text sowie Paragraphen der Knoten abfragbar
	\end{itemize}
\end{block}
\end{frame}
\begin{frame}
\frametitle{\LaTeX~- StyleChecker -- LanguageTool}
\begin{block}{\vspace*{-3ex}}
	\begin{itemize}
		\item Eintragung in der pom.xml 
		\item Einbindung als externe Bibliothek und über eine eigens verfasste Wrapper-Klasse
		\begin{itemize}
			\item als externe Bibliothek und 
			\item über Wrapper-Klasse
		\end{itemize}
		\item Suchen von Math-Umgebungen im Dokument, Überprüfung der Grammatik im Dokuments
	\end{itemize}
\end{block}
\end{frame}
\begin{frame}
\frametitle{\LaTeX~- StyleChecker -- Umsetzung der Regeln}
\begin{block}{\vspace*{-3ex}}
	\begin{itemize}
		\item Mit Hilfe des DocumentTrees
		\item Mit Hilfe von regulären Ausdrücken
		\begin{itemize}
			\item Erkennen von Sätzen und Abkürzungen
			\item Beispiel: Überschriften
		\end{itemize}
		\item Einsatz von regexr.com
		\item Zugriff auf die einzelnen Klassen über eine API-Klasse
	\end{itemize}
\end{block}
\end{frame}